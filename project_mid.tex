\documentclass[11pt]{article}

% Use Helvetica font.
%\usepackage{helvet}

%\usepackage[T1]{fontenc}
%\usepackage[sc]{mathpazo}
\usepackage{color}
\usepackage{amsmath,amsthm,amssymb,multirow,paralist}
%\renewcommand{\familydefault}{\sfdefault}

% Use 1/2-inch margins.
\usepackage[margin=0.8in]{geometry}
\usepackage{hyperref}

\begin{document}

\begin{center}
{\Large \textbf{Your Project Title}

\vspace{10pt}

Project Midterm Report}

\vspace{10pt}

\textit{Team Members: xxx1, xxx2, xxx3}
\end{center}

% \linethickness{1mm}\line(1,0){498}

%%%%%%%%%%%%%%%%%%%%%%%%%%%%%%%%%%%%%%%%%%%%%%%%%%%%%%%%%%%%%%%%%%%%%%%%%%%%%%%

%%%%%%%%%%%%%%%%%%%%%%%%%%%%%%%%%%%%%%%%%%%%%%%%%%%%%%%%%%%%%%%%%%%%%%%%%%%%%%%

\begin{abstract}
  Put your abstract here. It should be a brief summary of your
  task, motivation, algorithm, and results. Please keep it about
  150 words.
\end{abstract}

{\color{red} The length of your midterm report should be at least
3 pages not including references. Please replace contents with
your report contents, and then delete this paragraph.

Note that for groups with only one student, the requirement for
this midterm report length is 2 excluding references. You can
change the length of each part proportionally.}

\section{Introduction}

Use several paragraphs to introduce the background, machine
learning tasks, the dataset you use, your target machine learning
models, and your expected outcomes.

\section{Related Work}

Please briefly introduce some related works to your project. The
related work could be some papers related to your machine
learning algorithm.

Usually, this part accounts for 0.5 page. Here, you can list one
or two related works for midterm. The final version could be
finished in the final report.

\section{Apply machine learning to predict stock prices}

In this section, you will introduce your machine learning
algorithms used in this part.

If you are using existing methods on a \textbf{new dataset}. You
can talk about ``dataset collection'', ``data pre-processing'',
``Feature selection'', ``Choices of hyper-parameter'',
``Evaluation metrics'', etc.

If you propose some \textbf{new methods}. You can talk about
``Drawbacks of existing methods'', ``Your proposed methods'',
``Your models based on your methods'', etc.

These are possible subsections in this part. You don't need to
include all of them. I want to see the contents that you spend
time on and have some insights.

This part is one of the most important sections. The length of this
section should be about 1.5 pages.

\subsection{Dataset}
The dataset is collected from the Yahoo Finance website, which is a financial news and data website that provides various financial data,
including stock prices, market indices, exchange rates, and so on. 
We took the stock data of the top 10 food chains from the year 2021 to 2024. 
We use 80\% of the data as training data, and the most recent 20\% as test.

\subsection{Method}

So far, we use the well-know technical analysis indicators of the stock as our training features.
That  including: simple moving average (SMA),
convergence divergence moving average (MACD), and relative
strength index (RSI). The equations are shown below.

We calculate the these indicators of a time window of 20 days. The feature was created using the historical prices of the previous 
20 days, and the target is the price of the next day.

The ML models we have tried now include Xgboost and Elastic net regularization.

 the elastic net is a regularized regression method that linearly combines the L1 and L2 penalties of the lasso and ridge methods. 
 The quadratic penalty term makes the loss function strongly convex, and it therefore has a unique minimum.
 
 XGBoost (eXtreme Gradient Boosting) is a gradient-boosting machine learning algorithm. It is know for its speed and scalability.
 And it is quite flexible to adapt various datasets. So we take it in our experiment.

We are also exploring the Long Short Term Memory (LSTM) algorithm. For stock price prediction, LSTM network performance has
been greatly appreciated. It introduced the gate mechanism in RNN and the Attention Mechanism.
It store the historical information and the Attention-based Convolutional Neural Networks allows it to remember the useful information.
Thus LSTM is popular for predicting the time series problems and outperforms other ML methods in accuracies. 
On the other hand, for sure there is still lots of room to research the model and make more accurate prediction. 
So  


There are lots of interesting research to further improve the model accuracies 



The LSTM algorithm
has the ability to store historical information and is widely used in
stock price prediction (Heaton et al. 2016).
For stock price prediction, LSTM network performance has
been greatly appreciated when combined with NLP, which uses
news text data as input to predict price trends. In addition, there
are also a number of studies that use price data to predict price
movements (Chen et al. 2015), using historical price data in
addition to stock indices to predict whether stock prices will
increase, decrease or stay the same during the day (Di Persio and
Honchar, 2016), or compare the performance of the LSTM with
its own proposed method based on a combination of different
algorithms (Pahwa et al. 2017).
Zhuge et al. (2017) combine LSTM with Naiev Bayes method
to extract market emotional factors to improve predictive
performance. This method can be used to predict financial
markets on completely different time scales from other variables.
The sentiment analysis model is integrated with the LSTM time
series model to predict the stock?s opening price and the results
show that this model can improve the prediction accuracy.
Jia (2016) discussed the effectiveness of LSTM in stock price
prediction research and showed that LSTM is an effective method
to predict stock returns. The real-time wavelet transform was
combined with the LSTM network to predict the East Asian stock
index, which corrected some logic defects in previous studies.
Compared with the model using only LSTM, the combined model
can greatly improve the prediction degree and the regression error
is small. In


The evaluation criteria for the regression model is MSE and RSE, as given in Eq. \ref{eq:mse} and Eq\ref{eq:mape}.

MSE is calculated by taking the difference between the
predicted and actual values and squaring them,

MAPE is calculated by taking the absolute value of the difference
between the predicted and actual values and dividing it by the actual
value

This value is used to assess the
accuracy of the model and the closer it is to 1, the better the model



\section{Preliminary Results}

In this section, you will list some experimental results with
some comparisons with some baseline methods.

For example, you choose to use SVM on a new dataset. Possible
baselines would be linear perceptron and logistic regression by
using the same set of features.

In this midterm report, you can use a table to show some
preliminary results along with some discussions. This part should
be about 0.5 page.



\section{Future plan}

In this section, briefly discuss your plan based on current
achievements.

\section{References}

Please include your references here.


\end{document}
